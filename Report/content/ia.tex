\subsection{Literature review, state of the field}

\noindent Theoretical work has been done on the subject of using \textit{non-symbolic} methods for in-core fuel management, but very little implementation has been achieved \cite{anderson19}. The most reoccurring methods that appear in the literature are \textit{Particle Swarm Optimization}, \textit{fussy techniques}, \textit{tabu search} and \textit{cellular automata}, however those are still explicitly logic-based genetic-style algorithms, a new approach would be to directly train some neural networks or ants based pheromone mimetism. 
\subsection{IA and fuel management, a strategic approach}


\subsection{Machine-learning algorithms}
\begin{itemize}
    \item \textbf{CNN (Convolutional neural network)}\\
    From libraries such as \textit{TensorFlow} or \textit{OpenNN}, it is fairly easy to build a convolutional neural network. The main difficulty of this approach is to build a sensible database which has an effective conversion table to the weighting functions of the nodes of the net \cite{net}. One idea, which could be beneficial but hard to put in place would be to gather data on how the engineers themselves work on the tools. This is hard to do for two reasons, first as it is not really ethical to record what the worker does at all time, it is required to plan short sessions in advance where the work is recorded and the worker made completely aware. Secondly following that strategy more work has to be done on the data-structure on which it has to be recorded, and if this approach would be sufficient to have a database large enough to train the algorithm. \\
    
    Another approach would be to use LPO to create starting LP, then from further random shuffling and the large database available on autunite try to train the algorithm to get from generation to generation as close as the available database (and thus, training the algorithm for a long list of generated fictitious plans). The strength of this approach is that all the data is available. However we need to implement a generative planning system which is not an easy task, but can be done with traditional logic-based symbolic algorithm.
    
    \item \textbf{Ant based biomimetism} \\
    In \cite{ephraim19} is exposed another approach to genetic algorithm from the way ants find the most effective path in \textit{Traveling salesman} type of problem. The setback is that if this approach taken the system has to be implemented from scratch as this approach is problem-bound. It could however be an original direction. The strategy would be close to the second CNN one to train the "colony" of ants to find the most effective path to one of the configuration in the searchspace. See the complete description still in \cite{ephraim19}.
    
    \item \textbf{Particle Swarm Optimization} \\
    This method pops up often in the literature. This would be a sensible choice as it has been implemented for a lot of combinatory problems in the past, but no in in-core fuel management problems \cite{anderson19}. This has to be explored further.
\end{itemize}

\subsection{Strategies}
The following essential question is how can we use the previous algorithms to help the engineers. In all strategies, we use the same main idea : 
\begin{enumerate}
    \item \textbf{Heuristically improved swapping} : The idea is to implement a slight improvement to LPO after the fact. Panache will be the tool used to evaluate the quality of the LP. The NN will start with fully randomized symmetrical swapping. At each move, the quality function will give feedback on of it was a good decision or not. 
    \item \textbf{Post-LPO filtering} : LPO gives 500 possible LP, implementation of ML pattern recognition if well trained could be useful to very quickly make a hierarchy in all of them, from most probable solution to worse solution. 
    \item \textbf{Autunite pattern comparison} : 
\end{enumerate}

\section{Proof of work prototype}
\subsection{Conception}
\subsection{Results}
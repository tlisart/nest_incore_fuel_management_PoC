\section{Introduction}

\noindent The design and improvement of the loading pattern scheme in a PWR system relative to the refueling planning of a plant is a very complicated process. Two aspects of the engineering sciences have to be explored in the design of such a loading pattern. First the technical aspect relative to the RAMS studies (Reliability, Availability, Maintainability, and Safety) of the reactor, that is maintaining safety margins for controlling and shutting down the reactor. Second the economical aspects where at any point the power demand has to be reached as well as avoiding wastes. Luckily, through the 70' until now many optimizing methods have been developed in the area with the rise of many optimization algorithms. Two main techniques have now been used for a while : genetic algorithms which are quite powerful to give good answers to multi-variables optimization problems, and simulated annealing for finding a global optimum of a function. Two of those are meta-heuristic probabilistic methods and will be powerful tools in the conception of a loading pattern design. \\

\noindent Machine learning is not a new field, \textit{symbolic} machine learning has been worked on since the seventies and has given its load of results. However we witnessed since the beginning of the 2000 a boom around \textit{non-symbolic} machine learning systems, that is systems acting like a "Black box", where it is not possible to figure out its inner workings after training. The latter systems showed amazing results in fields such as image recognition, language recognition and generative text as well as some encouraging properties in combinatory problems and a variety of families of optimization problems (or near-optimal). That is why this first look into non-symbolic machine learning in loading pattern optimization is justified. \cite{anderson19}\\

\noindent As the task at hand is to find a place in the loading pattern design process where the addition of ML (Machine-Learning) code can improve the workflow of the engineer, it is clear that the process as it is done now have to be exposed : \\

\noindent The first step of design is building a first refueling plan based on various time constraints, knowing how long it takes for refueling, the workers availability, possible maintenances, meeting \textbf{Elia} energy demand and the period of the year, for example. When the schedule is mostly done, the engineer has to design a loading pattern respecting those parameters, where the \textbf{natural life} of the reactor with possible stretch-out up to 60 days are possible (at lower power output). To do so, tools are available to the engineer at Tractebel : database search algorithms are used to give a head start on the design, as well as genetic search algorithm using shuffling methods. However it is up to the engineer to tweak and rework what the tools gave. The search space in which the acceptable fuel configurations lays is so large (ant multi-dimensional) that the complete and ready design -even with the previously mentioned tools- requires a lot of manual work from the engineer \cite{ephraim19}. That is why non-symbolic machine learning to help this process is envisioned and looked after by Engie-Tractebel. \\

\noindent The goal of this project is thus to understand and explore what are the "Artificial Intelligence" methods available, what are the ones adapted to this problem and establish three strategies and how to implement them. A prototype is build on one of the proposed strategies. 

\begin{comment}
\noindent To be clear, the loading pattern design has to be completely defined from the local crayons fission parameters, which would be the reactivity, local burnup, global criticality, and local criticality. Moreover locally the cells in the reactor have preferential directions for neutron transport. That has to be taken into account when designing the loading pattern. Finally (and this is more specific to Tihange 2 reactor, but can be for all reactors), the heat peak delta f has to be minimized and non-local on the edges of the reactor to avoid a neutronic bombardment too high on the reflectors. \\

\noindent The loading pattern is usually completely managed by hand, after using internal tools based on genetic algorithm and database comparison. For designing a refined IA algorithm that would be able to further automatize the process both tools are available. In this paper three strategies have to be explored, seeing if indeed such an algorithm would be economical to use and develop. Of course the first step would then to unequivocally define the problem which has to be solved by the algorithm, making a list of constraints (physical, security norms, and economical constraints). We will see what algorithm or already used method in the industry would be the smart choice. Finally based on the three possibilities and after analysis of feasibility we will see what strategy can be prototyped within time constraints and resources available. \\
\end{comment}
